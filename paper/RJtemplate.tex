\title{\pkg{roxygen2}: In-Source Documentation for R}
\author{by Author One, Author Two and Author Three}

\maketitle

\abstract{
  Writing software documentation is hard work. \pkg{roxygen2}
  simplifies this task by coupling source code with documentation, and
  producing R compatible Rd files. Additionally, \pkg{roxygen2}
  automatizes the generation of R package infrastructure files. This
  paper describes the usage of \pkg{roxygen2}---its features and
  integration into the development workflow---with an illustrative
  example package.
}


%%%%%%%%%%%%%%%%%%%%%%%%%%%%%%%%%%%%%%%%%%%%%%%%%%%%%%%%%%%%%%%%%%%%%%
\section{Introduction\label{intro}}

\begin{quote}
  ``I believe that the time is ripe for significantly better
  documentation of programs, and that we can best achieve this by
  considering programs to be works of literature.''
\end{quote}
said \citeauthor{Knuth@1992} in his well-known book ``Literate
Programming'' \citep{Knuth@1992}, where he defined a methodology that
combines a programming language with a documentation language. Since
then, a lot of tools have been developed in the sense of this
term. Roughly, two different approaches have emerged: \textit{classical
  literate programming} systems like Sweave \citep{Leisch@2002} with
interleaving code and documentation chunks; and \textit{interface
  documentation} tools like Doxygen \citep{doxygen} and Javadoc
\citep{javadoc} with pseudo statements written as comments in the
source code and then processed to documentation. \pkg{roxygen2}
\citep{roxygen2} affiliates in the second approach with Doxygen and
Javadoc as inspirations. In addition to the common in-source
documentation of functions, methods, etc., R specific in-source
directives for namespace handling and source file collation are
available.


% Using Roxygen couples source code with documentation which simplifies
% holding the documentation consistent. All information concerning one
% logical part, like a function, class, source file or package, is
% centralized on one location, namely in front of this logical
% part. Additionally, Roxygen automatizes the generation of package
% infrastructure files (like the \file{NAMESPACE} and \file{DESCRIPTION}
% files) and allows the generation of call graphs.




%%%%%%%%%%%%%%%%%%%%%%%%%%%%%%%%%%%%%%%%%%%%%%%%%%%%%%%%%%%%%%%%%%%%%%
\section{Basics\label{basics}}

\begin{enumerate}
  \item Idea of roxygen (graph)
  \item How to write roxygen commands
  \item Roclets
  \item \code{roxygenize}
\end{enumerate}



%%%%%%%%%%%%%%%%%%%%%%%%%%%%%%%%%%%%%%%%%%%%%%%%%%%%%%%%%%%%%%%%%%%%%%
\section{In-source documentation\label{doc}}


\begin{enumerate}
  \item Rd roclet
\end{enumerate}




%%%%%%%%%%%%%%%%%%%%%%%%%%%%%%%%%%%%%%%%%%%%%%%%%%%%%%%%%%%%%%%%%%%%%%
\section{Further in-source specifications\label{doc}}

\begin{enumerate}
  \item Collate and namespace roclet
\end{enumerate}



%%%%%%%%%%%%%%%%%%%%%%%%%%%%%%%%%%%%%%%%%%%%%%%%%%%%%%%%%%%%%%%%%%%%%%
\section{\pkg{roxygen2} in the wild\label{doc}}

\begin{enumerate}
  \item \pkg{stringr}, \pkg{ggplot2}, \pkg{archetypes}
  \item see \url{http://crantastic.org/tags/roxygen2} which lists all
    packages with \code{Build-Depends: roxygen2} set in the
    \code{DESCRIPTION} file.
\end{enumerate}
 


%%%%%%%%%%%%%%%%%%%%%%%%%%%%%%%%%%%%%%%%%%%%%%%%%%%%%%%%%%%%%%%%%%%%%%
\section{Outlook\label{outlook}}




%%%%%%%%%%%%%%%%%%%%%%%%%%%%%%%%%%%%%%%%%%%%%%%%%%%%%%%%%%%%%%%%%%%%%%
\section*{Acknowledgment}

\pkg{roxygen}, the predecessor of \pkg{roxygen2}, was a Google Summer
of Code Project in 2008 \citep{roxygen}.





% Introductory section which may include references in parentheses, say
% \citep{R:Ihaka+Gentleman:1996} or cite a reference such as
% \citet{R:Ihaka+Gentleman:1996} in the text.

% \section{Section title in sentence case}

% This section may contain a figure such as Figure \ref{figure:onecolfig}.

% \begin{figure}
% \vspace*{.1in}
% \framebox[\textwidth]{\hfill \raisebox{-.45in}{\rule{0in}{1in}}
%                       A picture goes here \hfill}
% \caption{\label{figure:onecolfig}
% A normal figure only occupies one column.}
% \end{figure}

% \section{Another section}

% There will likely be several sections, perhaps including code snippets, such
% as
% \begin{example}
%   x <- 1:10
%   result <- myFunction(x)
% \end{example}

% \section{Summary}

% This file is only a basic article template. For full details of \emph{The R Journal}
% style and information on how to prepare your article for submission, see the
% \href{http://journal.r-project.org/latex/RJauthorguide.pdf}{Instructions for Authors}.

\bibliography{references}

\address{Author One\\
  Affiliation\\
  Address\\
  Country}\\
\email{author1@work}

\address{Author Two\\
  Affiliation\\
  Address\\
  Country}\\
\email{author2@work}

\address{Author Three\\
  Affiliation\\
  Address\\
  Country}\\
\email{author3@work}
